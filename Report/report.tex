% ****** Start of file apssamp.tex ******
%
%   This file is part of the APS files in the REVTeX 4.2 distribution.
%   Version 4.2a of REVTeX, December 2014
%
%   Copyright (c) 2014 The American Physical Society.
%
%   See the REVTeX 4 README file for restrictions and more information.
%
% TeX'ing this file requires that you have AMS-LaTeX 2.0 installed
% as well as the rest of the prerequisites for REVTeX 4.2
%
% See the REVTeX 4 README file
% It also requires running BibTeX. The commands are as follows:
%
%  1)  latex apssamp.tex
%  2)  bibtex apssamp
%  3)  latex apssamp.tex
%  4)  latex apssamp.tex
%
\documentclass[%
%%%% Choose between preprint (1-column) and reprint (2-column) here
%preprint,
reprint,
%%%%
%superscriptaddress,
%groupedaddress,
%unsortedaddress,
%runinaddress,
%frontmatterverbose, 
%preprintnumbers,
%nofootinbib,
%nobibnotes,
%bibnotes,
 amsmath,amssymb,
 aps,
%pra,
%prb,
%rmp,
%prstab,
%prstper,
%floatfix,
]{revtex4-2}

\usepackage{graphicx}% Include figure files
\usepackage{dcolumn}% Align table columns on decimal point
\usepackage{bm}% bold math
%\usepackage{hyperref}% add hypertext capabilities
%\usepackage[mathlines]{lineno}% Enable numbering of text and display math
%\linenumbers\relax % Commence numbering lines

%\usepackage[showframe,%Uncomment any one of the following lines to test 
%%scale=0.7, marginratio={1:1, 2:3}, ignoreall,% default settings
%%text={7in,10in},centering,
%%margin=1.5in,
%%total={6.5in,8.75in}, top=1.2in, left=0.9in, includefoot,
%%height=10in,a5paper,hmargin={3cm,0.8in},
%]{geometry}

\begin{document}

\preprint{APS/123-QED}

\title{Quantum Simulation of the $H_2$ Molecule}
\thanks{A footnote to the article title}%

\author{Robert Negrin}
\affiliation{
 University of California \\
 Los Angeles, California% with \\
}%

\author{Saleh Naghdi}
\affiliation{
 University of California \\
 Los Angeles, California% with \\
}%

\author{Yarin Heffes}
\affiliation{
 University of California \\
 Los Angeles, California% with \\
}%

\author{Manvi Agrawal}
\affiliation{
 University of California \\
 Los Angeles, California% with \\
}%

\date{\today}% It is always \today, today,
             %  but any date may be explicitly specified

\begin{abstract}
    \section*{Abstract}
    Our project focuses on simulating the behavior and energy of the H2 molecule in various quantum scenarios.
    Through this exploration, we've captured key insights into the ground state energy and effects of different perturbations.


\begin{description}
    \begin{enumerate}
        \item[Initial Simulation and Ground State]

        \begin{itemize}
            \item H2 Molecule: Simulated the H2 molecule using a quantum approach.
            
            \item VQE (Variational Quantum Eigensolver): Determined the ground state energy of the H2 molecule.
        \end{itemize}

        
        \item[Trotterisation and Perturbation]
            \begin{itemize}
                \item  Trotterisation: Applied the Trotterisation method to see the changes in energy due to a perturbation.
                \item Time-Dependent Perturbation: Introduced a time-dependent perturbation to study its effects.
            \end{itemize}
        
        \item [Application of Magnetic Field]
            \begin{itemize}
                \item Magnetic Field Orientation: Applied a magnetic field that transitions from the X to the Z axis.
                \item Analysis: Observed and documented the effect of the changing magnetic field on the ground state energy of the H2 molecule.
            \end{itemize}

    \end{enumerate}
\end{description}
\end{abstract}

%\keywords{Suggested keywords}%Use showkeys class option if keyword
                              %display desired
\maketitle

%\tableofcontents

\section{\label{sec:level1}Initial Simulation and Ground State
H2 Molecule}

\subsection{\label{sec:level2} Simulated the H2 molecule}
We utilized Qiskit's `PySCFDriver`' to define the $H_2$ molecule. The electronic structure of the 
molecule is initially described in terms of fermionic creation and annihilation operators. \newline

Now, transform this fermionic Hamiltonian into a qubit (spin) Hamiltonian suitable for quantum computation.
For our simulation we have used Jordan-Wigner Transformation which provides a direct mapping of fermions to qubits.
Bravyi-Kitaev Transformation is another more efficient mapping especially for systems with local interactions, but its more complex in form.



\subsection{\label{sec:level2}VQE (Variational Quantum Eigensolver)}
Now, we can use Variational Quantum Eigensolver(VQE) to determine the ground state of the hydrogen atom.

For this we used `TwoLocal`' ansatz in Qiskit. It is is a quantum circuit template used in VQE, and alternates between $R_y$ and $C_Z$ gates
We used this structure since it provides a balance between expressiveness and computational efficiency, making it a popular choice for approximating quantum states.


\section{\label{sec:level1}Evolution of H2 molecule}

Now, to find the time evolution of H2 molecule, we can use a couple of approaches:

\begin{enumerate}
    \item Time evolution by matrix exponentation(Naive): We can multiply the Hamiltonians at small $dt$ intervals. However, its a complex process
    \item Trotterisation: It helps to approximate the Hamiltonian, thereby making it more computationally efficient.
    \item Magnus Expansion: Another way to approximate a Time dependent Hamiltonian
\end{enumerate}

\subsection{\label{sec:level2} Use  to find the time evolution}
We utilized Qiskit's `SuzukiTrotter` to evolve the $H_2$ molecule in time.

For this we first used a time independant Hamiltonian and first order Trotterisation, and we observe that Trotterisation matches the exact energy. \newline

Next, we apply $H_{\text{perturbation}}$ as stated below, to see how well Trotterisation is able to approximate the time evolution of Hamiltonian. \\
$$
H_{\text{perturbation}} = -h \left( \sum_{i=0}^{N-1} \sin(\alpha) Z_i + \cos(\alpha) X_i \right)
$$

We observe that for first order Trotterisation, the variations are significant. But as we increase the order, from 1 $\rightarrow$ 2 $\rightarrow$ 4, 
the accuracy energy evolution fit gets better.

\end{document}
%
% ****** End of file apssamp.tex ******
